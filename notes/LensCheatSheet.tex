% This file was converted to LaTeX by Writer2LaTeX ver. 1.0.2
% see http://writer2latex.sourceforge.net for more info
\documentclass[a4paper]{article}
\usepackage[ascii]{inputenc}
\usepackage[T1]{fontenc}
\usepackage[english]{babel}
\usepackage{amsmath}
\usepackage{amssymb,amsfonts,textcomp}
\usepackage{color}
\usepackage{array}
\usepackage{hhline}
\usepackage{hyperref}
\hypersetup{pdftex, colorlinks=true, linkcolor=blue, citecolor=blue, filecolor=blue, urlcolor=blue, pdftitle=, pdfauthor=, pdfsubject=, pdfkeywords=}
% List styles
\newcommand\liststyleWWNumiii{%
\renewcommand\labelitemi{${\bullet}$}
\renewcommand\labelitemii{${\circ}$}
\renewcommand\labelitemiii{${\blacksquare}$}
\renewcommand\labelitemiv{${\bullet}$}
}
\newcommand\liststyleWWNumv{%
\renewcommand\labelitemi{${\bullet}$}
\renewcommand\labelitemii{${\circ}$}
\renewcommand\labelitemiii{${\blacksquare}$}
\renewcommand\labelitemiv{${\bullet}$}
}
\newcommand\liststyleWWNumi{%
\renewcommand\labelitemi{${\bullet}$}
\renewcommand\labelitemii{${\circ}$}
\renewcommand\labelitemiii{${\blacksquare}$}
\renewcommand\labelitemiv{${\bullet}$}
}
\newcommand\liststyleWWNumii{%
\renewcommand\labelitemi{${\bullet}$}
\renewcommand\labelitemii{${\circ}$}
\renewcommand\labelitemiii{${\blacksquare}$}
\renewcommand\labelitemiv{${\bullet}$}
}
\newcommand\liststyleWWNumiv{%
\renewcommand\labelitemi{${\bullet}$}
\renewcommand\labelitemii{${\circ}$}
\renewcommand\labelitemiii{${\blacksquare}$}
\renewcommand\labelitemiv{${\bullet}$}
}
% Page layout (geometry)
\setlength\voffset{-1in}
\setlength\hoffset{-1in}
\setlength\topmargin{0in}
\setlength\oddsidemargin{1in}
\setlength\textheight{8.732699in}
\setlength\textwidth{6.2681in}
\setlength\footskip{0.0cm}
\setlength\headheight{1in}
\setlength\headsep{0.9602in}
% Footnote rule
\setlength{\skip\footins}{0.0469in}
\renewcommand\footnoterule{\vspace*{-0.0071in}\setlength\leftskip{0pt}\setlength\rightskip{0pt plus 1fil}\noindent\textcolor{black}{\rule{0.0\columnwidth}{0.0071in}}\vspace*{0.0398in}}
% Pages styles
\makeatletter
\newcommand\ps@Standard{
  \renewcommand\@oddhead{}
  \renewcommand\@evenhead{\@oddhead}
  \renewcommand\@oddfoot{}
  \renewcommand\@evenfoot{}
  \renewcommand\thepage{\arabic{page}}
}
\makeatother
\pagestyle{Standard}
\title{}
\author{}
\date{}
\begin{document}
\clearpage\setcounter{page}{1}\pagestyle{Standard}

The initial insight of lenses was to pack up a getter and a setter:


\bigskip

\ \ \ \ data Lens s a = Lens (s -{\textgreater} a) (a -{\textgreater} s
-{\textgreater} s)


\bigskip

This type is actually called Lens{\textquoteright} in this package.
\ Its capability corresponds to the getting and setting of the fields
of a type with a single constructor. \ For setting a simpler type like
a newtype it is more powerful than necessary, and this and similar
cases are handled by the Iso type:


\bigskip

\ \ \ data Iso s a = Iso (s -{\textgreater} a) (a -{\textgreater} s)


\bigskip

On the other hand, lenses are not powerful enough to deal with a type
that has multiple constructors, for which the Prism type is provided:


\bigskip

\textbf{IsA heirarchy}


\bigskip

\liststyleWWNumiii
\begin{itemize}
\item Every Prism is a Traversal
\item Every Iso is a Prism
\item Every Iso is a Lens
\item Every Lens is a Setter
\item Every Lens is a Getter
\item Every Getter is a Fold
\end{itemize}

\bigskip

\textbf{Basic Functions for Applying}


\bigskip

\liststyleWWNumv
\begin{itemize}
\item view {}- Extract a value for a Getter, or a Monoid for a Fold or
Traversal
\item use - Use the target of a Lens, Iso, or Getter in the current
state, or use a summary of a Fold or Traversal that points to a
monoidal value.
\item listening - Writer monad
\item over - modify target (Setter)
\item set - Replace the target of a Lens or all of the targets of a
Setter or Traversal with a constant value. (Setter)
\end{itemize}

\bigskip

\textbf{Basic functions for building}


\bigskip

\liststyleWWNumi
\begin{itemize}
\item lens{\textquoteright}
\item iso{\textquoteright}
\item prism{\textquoteright}
\item re - Turn a Prism or Iso around to build a Getter. (Review)
\item to - Build an (index-preserving) Getter from an arbitrary Haskell
function. (Getter)
\item from - invert an Iso (Iso)
\item mapped - This Setter can be used to map over all of the values in
a Functor.
\item singular
\end{itemize}

\bigskip

\textbf{Identities}

\bigskip

\liststyleWWNumii
\begin{itemize}
\item view . to ${\equiv}$ id
\item from (from l) ${\equiv}$ l
\item under ${\equiv}$ over . from
\item f . from f ${\equiv}$ id\newline
from f . f ${\equiv}$ id
\item review ${\equiv}$ view . re
\item review . unto ${\equiv}$ id
\item 
\bigskip
\end{itemize}

\bigskip

\textbf{Name conventions}


\bigskip

\liststyleWWNumiv
\begin{itemize}
\item Prefix {\textquotedblleft}i{\textquotedblright} - indexed variant
\item Suffix {\textquotedblleft}s{\textquotedblright} -
{\textquotedblleft}function of{\textquotedblright} variant
\item Suffix single quote - simplified variant
\end{itemize}
\end{document}
